\part{利用者向け仕様書}
\chapter{利用者向け仕様書}

\section{概要}
このシステムは、日本の都道府県データをWebブラウザ上で閲覧・管理できるアプリケーションである。
一覧表示、詳細情報の確認に加え、新しい都道府県データの追加、既存データの編集および削除を行うことができる。


\section{使用できる機能}
このシステムでは以下の機能を使用することができる。
\begin{itemize}
    \item 都道府県データの一覧表示
    \item 各都道府県の詳細情報(人口、面積、県庁所在地など)の表示
    \item 新規データの登録
    \item データの編集
    \item データの削除
\end{itemize}

\section{画面構成と操作方法}

\subsection{起動画面(トップページ)}
システムにアクセスすると、各管理システムへのリンクが表示されるトップページが開く。
「都道府県一覧表示システム」のリンクをクリックすることで、一覧画面へ遷移する。

\begin{figure}[H]
    \centering
    \includegraphics[width=8cm]{fig/user/top_page.png}
    \caption{トップページ}
    \label{fig:top_page}
\end{figure}

\subsection{一覧表示}
登録されている都道府県データが表形式で表示される。
\begin{itemize}
    \item 詳細リンク: 都道府県名をクリックすると、詳細画面へ移動する。
    \item 追加ボタン: 画面上部のボタンから新規作成画面へ移動する。
\end{itemize}

\begin{figure}[H]
    \centering
    \includegraphics[width=8cm]{fig/user/pref_show.png}
    \caption{一覧表示}
    \label{fig:list_page}
\end{figure}

\subsection{詳細表示}
選択した都道府県の詳しい情報(コード、面積、人口、県庁所在地、地方区分)を確認できる。
この画面から、情報の「編集」または「削除」を行う画面へ移動できる。

\begin{figure}[H]
    \centering
    \includegraphics[width=8cm]{fig/user/detail.png}
    \caption{詳細表示}
    \label{fig:detail_page}
\end{figure}

\subsection{データ追加}
新規作成画面にて、必要な情報を入力し「登録」ボタンを押すことで、新しいデータをリストに追加できる。
「登録して新たに作成」を選択すると、連続してデータを入力することが可能である。
% 解説: app_system.js の処理 に基づく

\subsection{データ編集}
詳細画面の「編集」ボタンから、既存のデータを修正できる。内容は即座に一覧および詳細画面に反映される。

\subsection{データ削除}
詳細画面の「削除」ボタンを押すと、確認画面が表示される。
削除対象のデータを確認し、問題なければ実行することでデータが削除される。

\section{Q\&A (よくある質問)}

\begin{description}
    \item[Q. 登録したデータが消えてしまった] \hfill \\
    A. システムの仕様上、サーバーを停止または再起動すると、登録・編集した内容はすべて初期状態(インストール時のデータ)に戻ります。

    \item[Q. スマートフォンから利用できるか?] \hfill \\
    A. Webブラウザ(Chrome, Safari等)が搭載された端末であれば、PCと同様に利用可能です。
    % 解説: レスポンシブかどうかはCSS次第ですが、基本機能としては利用可能です。
\end{description}
