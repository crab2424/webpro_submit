\newcommand{\myfigure}[2]{
\begin{figure}[H]
    \centering
    \includegraphics[width=12cm]{fig/user/#1.png}
    \caption{#2}
    \label{fig:user_#1}
\end{figure}
}

\part{利用者向け仕様書}
\chapter{利用者向け仕様書}

\section{概要}
このシステムは、Webブラウザ上で一覧データを閲覧・管理できるアプリケーションである。
一覧表示、詳細情報の確認に加え、新しいデータの追加、既存データの編集および削除を行うことができる。
一覧表示できるシステムは都道府県、星座、元素の合計3つであるが、この仕様書では都道府県システムを代表して説明する。

\section{使用できる機能}
このシステムでは、以下の機能を利用することができる。
\begin{itemize}
    \item 都道府県データの一覧表示
    \item 各都道府県の詳細情報(人口、面積、県庁所在地など)の表示
    \item 新規データの登録
    \item データの編集
    \item データの削除
\end{itemize}

\section{画面構成と操作方法}

\subsection{起動画面(トップページ)}
このWebアプリケーションは、以下のURLをクリックまたはブラウザのリンクに貼り付けてアクセスすることで起動する。
\begin{center}
    \url{http://localhost:8080}
\end{center}

システムにアクセスすると、図\ref{fig:user_top_page}のような各管理システムへのリンクが表示されるトップページが開く。
「都道府県一覧表示システム」のリンクをクリックすることで、一覧画面へ遷移する。
\myfigure{top_page}{トップページ}


\subsection{一覧表示}
図\ref{fig:user_show}のように、登録されている都道府県データが表形式で表示される。都道府県名をクリックすると、詳細画面へ移動する。
また、画面下部の「追加」リンクから新規作成画面へ移動し、「トップへ戻る」リンクでトップページに戻ることができる。

\myfigure{show}{一覧表示画面}


\subsection{詳細表示}

選択した都道府県の詳しい情報(コード、面積、人口、県庁所在地、地方区分)を確認できる。
図\ref{fig:user_detail}の画面から、情報の「編集」、「削除」を行う画面への移動、または一覧表示に戻ることができる。
\myfigure{detail}{詳細表示}


\subsection{データ追加}

図\ref{fig:user_new}の新規作成画面にて、必要な情報を入力し「登録」ボタンを押すことで、新しいデータをリストに追加できる。数値の入力欄には、値のみを入力し、単位は自動的に付与される。「登録\&追加」を選択すると、連続してデータを入力することが可能である。
\myfigure{new}{新規作成画面}

図\ref{fig:user_add}のように入力してから「登録」ボタンを押すと、図\ref{fig:user_added}のように追加したデータと共に一覧表示画面に戻る。
\myfigure{add}{入力例}
\myfigure{added}{新規作成後}


\subsection{データ編集}

図\ref{fig:user_edit}の詳細画面の「編集」ボタンから、既存のデータを修正できる。内容は即座に一覧および詳細画面に反映される。
\myfigure{edit}{編集画面}

図\ref{fig:user_edit_update}のように入力してから「更新」ボタンを押すと、図\ref{fig:user_changed}のように編集したデータと共に詳細画面に戻る。
\myfigure{edit_update}{編集中}
\myfigure{changed}{編集後}


\subsection{データ削除}
図\ref{fig:user_check}のように、詳細画面の「削除」ボタンを押すと、確認画面が表示される。
削除対象のデータを確認し、問題なければ実行することでデータが削除される。

\myfigure{check}{削除確認画面}

「削除」ボタンを押すと、図\ref{fig:user_deleted}のように、そのデータの一覧表示画面に戻る。

\myfigure{deleted}{削除完了画面}

\section{Q\&A}

\begin{description}
    \item[Q. 登録したデータが消えてしまった] \hfill \\
    A. システムの仕様上、サーバーを停止または再起動すると、登録・編集した内容はすべて初期状態に戻ります。

    \item[Q. スマートフォンから利用できるか?] \hfill \\
    A. 原則としてPCからの利用を想定しています。
\end{description}
