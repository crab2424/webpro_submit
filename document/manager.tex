\part{管理者向け仕様書}
\chapter{管理者向け仕様書}

\section{概要}
本仕様書は,Node.jsおよびテンプレートエンジンEJSを用いたWebアプリケーションの仕様書である.
本システムは,サーバーサイドで都道府県データを管理し,EJSを用いて動的にHTMLを生成・表示する.
データベースの利用は行わず,サーバープロセスのメモリ上(変数)でデータを保持・操作することを前提とする.
なお,本システムはmacOS環境で開発したものであるため,macOSを基準とした管理方法を示す.

\section{インストール方法}

\subsection{動作環境}

本システムの動作環境およびインストール手順を以下に示す.
\begin{itemize}
    \item OS: Windows, macOS, または Linux
    \item サーバー実行環境: node.js (開発時バージョン v24.12.0)
    \item パッケージマネージャー: npm, nodebrew, homebrew
\end{itemize}

\subsection{セットアップ手順}

\subsubsection{homebrewのインストール}
JavaScriptを動作させるにあたって,macOSではhomebrewを用いてnodebrewをインストールする必要がある.まず,ターミナル上で以下のコマンドを実行する.その際,ターミナル上でパスワードを求められるので,パスワードを入力する.
\begin{Verbatim}[fontsize=\scriptsize]
$ /bin/bash -c "$(curl -fsSL https://raw.githubusercontent.com/Homebrew/install/HEAD/install.sh)"
\end{Verbatim}
次に,以下の2つのコマンドを順に実行する.
\begin{Verbatim}[fontsize=\small]
( echo; echo 'eval "$(/opt/homebrew/bin/brew shellenv)"') >> ~/.zprofile
eval "$(/opt/homebrew/bin/brew shellenv)"
\end{Verbatim}

\subsubsection{nodebrewのインストール}

ターミナルにて,以下の4つのコマンドを1つずつ順に実行してnodebrewをインストールする.

\begin{verbatim}
$ brew install nodebrew
$ nodebrew setup
$ echo 'export PATH=$HOME/.nodebrew/current/bin:$PATH' >> ~/.zshrc
$ source ~/.zshrc
\end{verbatim}


\subsubsection{node.jsのインストール}
ターミナルにて,以下のコマンドを1つずつ順に実行してnode.jsをインストールする.

\begin{verbatim}
$ nodebrew install stable
$ nodebrew ls
\end{verbatim}
これにより,現在のバージョンが表示されるため,それに従って以下のコマンドを実行する.
\begin{verbatim}
$ nodebrew use v24.12.0
$ npm install -g npm
\end{verbatim}


\section{起動・終了方法}

\subsection{ソースコード}

以下のURLにソースコードを添付する.\\
\url{https://github.com/crab2424/webpro_submit}\\

リポジトリページのcodeボタンから,Download ZIPを選択することで,ZIPファイルをダウンロードすることができる.

\subsection{起動方法}
まず,ソースコードが配置されたディレクトリ(webpro\_submit/app)に移動する.

\begin{verbatim}
$ cd webpro_submit/app
\end{verbatim}

次に,以下のコマンドを実行してサーバーを起動する.

\begin{verbatim}
$ node app_system.js
\end{verbatim}

起動に成功すると,コンソールに以下のメッセージが表示される.
\begin{verbatim}
Example app listening on port 8080!
\end{verbatim}

その後,Webブラウザで以下のURLにアクセスすると,トップページが表示される.
\begin{center}
    \url{http://localhost:8080/}
\end{center}

\subsection{終了方法}
サーバーを実行しているターミナルにおいて,以下のキーを入力してプロセスを終了する.
\begin{center}
    Ctrl + C
\end{center}


\section{トラブルシューティング}

\subsection{起動できない場合}
\begin{itemize}
    \item エラー: Address already in use \\
    ポート8080が他のプロセスで使用されている可能性がある.他のNode.jsプロセスを終了するか,PCを再起動してから再度実行すること.
    \item エラー: Cannot find module \\
    npm install が正しく実行されていない可能性がある.再度インストール手順を確認すること.
    また,現在のディレクトリが正しくない可能性がある.ディレクトリがwebpro\_submit/appであるか確認すること.
\end{itemize}

\section{既知の不具合・制限事項}
本システムは学習用アプリケーションとしての仕様上,以下の制限事項が存在する.
\begin{itemize}
    \item データの非永続性: データベースを使用せず,サーバーのメモリ上(変数)でデータを管理しているため,サーバーを再起動または終了すると,追加・編集・削除したデータは初期状態にリセットされる.
\end{itemize}
