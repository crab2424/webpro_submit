\part{管理者向け仕様書}
\chapter{管理者向け仕様書}

\section{概要}
本仕様書は,Node.jsおよびテンプレートエンジンEJSを用いたWebアプリケーションの仕様書である.
本システムは,サーバーサイドで都道府県データを管理し,EJSを用いて動的にHTMLを生成・表示する.
データベースの利用は行わず,サーバープロセスのメモリ上(変数)でデータを保持・操作することを前提とする.

\section{インストール方法}
本システムの動作環境およびインストール手順を以下に示す.

\subsection{動作環境}
\begin{itemize}
    \item OS: Windows, macOS, または Linux
    \item ランタイム: Node.js (推奨バージョン: LTS版)
    \item パッケージマネージャ: npm
\end{itemize}

\subsection{セットアップ手順}
ターミナル(コマンドプロンプト)にて,ソースコードが配置されたディレクトリ(webpro\_submit)に移動し,以下のコマンドを実行して必要なライブラリ(Express, EJS)をインストールする.

\begin{verbatim}
$ cd webpro_submit
$ npm install
\end{verbatim}


\section{起動・終了方法}

\subsection{起動方法}
以下のコマンドを実行してサーバーを起動する.

\begin{verbatim}
$ node app/app_system.js
\end{verbatim}


起動に成功すると,コンソールに以下のメッセージが表示される.
\begin{verbatim}
Example app listening on port 8080!
\end{verbatim}


その後,Webブラウザで以下のURLにアクセスする.
\begin{center}
    \url{http://localhost:8080/}
\end{center}

\subsection{終了方法}
サーバーを実行しているターミナルにおいて,以下のキーを入力してプロセスを終了する.
\begin{center}
    Ctrl + C
\end{center}

\section{トラブルシューティング}

\subsection{起動できない場合}
\begin{itemize}
    \item \textbf{エラー: Address already in use} \\
    ポート8080が他のプロセスで使用されている可能性がある.他のNode.jsプロセスを終了するか,PCを再起動してから再度実行すること.
    \item \textbf{エラー: Cannot find module} \\
    npm install が正しく実行されていない可能性がある.再度インストール手順を確認すること.
\end{itemize}

\section{既知の不具合・制限事項}
本システムは学習用アプリケーションとしての仕様上,以下の制限事項が存在する.
\begin{itemize}
    \item \textbf{データの非永続性}: データベースを使用せず,サーバーのメモリ上(変数)でデータを管理しているため [cite: 27],サーバーを再起動または終了すると,追加・編集・削除したデータは初期状態にリセットされる.
\end{itemize}
