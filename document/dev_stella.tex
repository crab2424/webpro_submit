\chapter[開発者向け仕様書:星座一覧表示システム]{\texorpdfstring{開発者向け仕様書:\\星座一覧表示システム}{開発者向け仕様書:星座一覧表示システム}}

\section{概要}
本仕様書は,Node.jsおよびテンプレートエンジンEJSを用いた「星座一覧表示システム」の設計仕様書である.
本システムは,サーバーサイドで星座データを管理し,EJSを用いて動的にHTMLを生成・表示する.
データベースの利用は行わず,サーバープロセスのメモリ上(変数)でデータを保持・操作することを前提とする.

\section{データ管理}

\subsection{データ構造}
サーバー内の配列変数で管理するデータ構造は表\ref{tab:schema}の通りに構成する.

\begin{table}[H]
    \centering
    \caption{星座データ構造}
    \label{tab:schema}
    \begin{tabular}{c|c|c}
        \toprule
        プロパティ名 & データ型 & 説明 \\
        \midrule
        id & Number & 一意な識別子 \\
        name & String & 星座名 \\
        en & String & 英語表記 \\
        shape & String & 星座の形 \\
        height & String & 高度($^\circ$) \\
        star & String & 代表する星 \\
        season & String & 季節 \\
        \bottomrule
    \end{tabular}
\end{table}

idを除くすべてのデータは入力フォームで作成,編集が可能である.idはデータ配列の長さに基づいて自動採番するため,作成時に入力不要である.

\subsection{表示における制約事項}
本システムでは,データidと配列インデックスの整合性を保つため,以下の仕様を採用する.

\begin{itemize}
    \item idとインデックスの一致: 配列のインデックスとidを一致させるため,配列のインデックス0番目にはダミーデータを作成し,システム上では非表示とする.
    \item 一覧表示の処理: idが1番のデータから表示させるため,EJSテンプレート側で配列操作(data.slice(1))を用いて表示する.
\end{itemize}

\section{ディレクトリ構成}

本システムは,図\ref{fig:stella_dir}に示すディレクトリ構造に従って,使用するファイルを配置する.

\begin{figure}[H]
    \centering
    \begin{verbatim}
webpro_submit/
    |- app/
        |- app_system.js            (メインロジック・データ変数保持)
        |- public/                  (静的ファイル)
            |- stella_new.html        (新規作成フォーム)
        |- views/                   (EJSテンプレート)
            |- stella/                (星座システム)
                |- stella_check.ejs   (削除確認画面)
                |- stella_detail.ejs  (詳細表示画面)
                |- stella_edit.ejs    (編集フォーム)
                |- stella.ejs         (一覧表示画面)
            |- landing.ejs          (トップページ)
\end{verbatim}
    \caption{ディレクトリ構成}
    \label{fig:stella_dir}
\end{figure}

\section{HTTPメソッドとルーティング}
本システムにおける各URLとHTTPメソッド,および対応する処理を表\ref{tab:routing}に定義する.

\begin{table}[H]
    \centering
    \caption{ルーティング一覧}
    \label{tab:routing}
    \begin{tabular}{llll}
        \toprule
        機能 & メソッド & パス (URL) & 対応ビュー\\
        \midrule
        トップページ & GET & / & views/landing.ejs \\
        一覧表示 & GET & /stella & views/stella/stella.ejs \\
        新規作成フォーム & GET & /stella/create & public/stella\_new.html \\
        詳細表示 & GET & /stella/:id & views/stella\_detail.ejs \\
        編集フォーム & GET & /stella/edit/:id & views/stella\_edit.ejs \\
        削除確認 & GET & /stella/check/:id & views/stella\_check.ejs \\
        \midrule
        新規データ作成 & POST & /stella & 処理後一覧を表示\\
        新規データ作成& POST & /stella/create & 処理後新規作成へリダイレクト \\
        データ更新 & POST & /stella/update/:id & 処理後詳細ページを表示 \\
        データ削除 & GET & /stella/delete/:id & 処理後一覧へリダイレクト \\
        \bottomrule
    \end{tabular}
\end{table}

\section{ページ遷移}

本システムにおける画面間の遷移を図\ref{fig:flowchart}に示す.なお,本システムのページには戻るリンクを配置するため,一覧表示ページ及び詳細表示ページに直接遷移することが可能である.

\vspace{1cm}
% フローチャート配置場所
\begin{figure}[H]
    \centering
    \includegraphics[width=\textwidth]{fig/dev_stella/flowchart-1.pdf}
    \caption{画面遷移フローチャート}
    \label{fig:flowchart}
\end{figure}


\section{各機能・リソース詳細}

\subsection{トップページ}
\begin{itemize}
    \item URL: /
    \item 処理: views/landing.ejsを表示する.
    \item 要素: 各システムへのリンク
\end{itemize}

\subsection{一覧表示機能}
\begin{itemize}
    \item URL: /stella
    \item 処理: サーバー変数の全データをEJSに渡し,for文を用いてテーブル形式で表示する.
    \item 要素: 各行ごとの要素の名前に対応する詳細リンク,追加ボタン
\end{itemize}

\subsection{新規作成機能}
\begin{itemize}
    \item フォーム: GET /stella/create
        \begin{itemize}
            \item 処理: 表\ref{tab:schema}のidを除くすべてのプロパティを入力フィールドとして表示する.
            \item 送信先: POST /stella
            \item 要素: 各プロパティに対応する入力フィールド,登録ボタン,登録後新規作成ボタン
        \end{itemize}
    \item 作成処理: POST /stella
        \begin{itemize}
            \item リクエストボディから値を取得.
            \item 新しいidを採番し,サーバー変数(配列)にpushする.
            \item 新規作成したデータの内容をサーバーのターミナルにログ出力する.
            \item 処理完了後,一覧画面(/stella)へリダイレクトする.
        \end{itemize}
\end{itemize}

\subsection{詳細表示機能}
\begin{itemize}
    \item URL: /stella/:id
    \item 処理: URLパラメータのidに基づき配列を検索し,対象データを表示する.
    \item 要素: 編集ボタン,削除ボタン,一覧に戻るリンク
\end{itemize}

\subsection{編集・更新機能}
\begin{itemize}
    \item フォーム: GET /stella/edit/:id
        \begin{itemize}
            \item 処理: 対象データを検索し,value属性に現在の値を埋め込んで表示する.
            \item 送信先: POST /stella/update/:id
        \end{itemize}
    \item 更新処理: POST /stella/update/:id
        \begin{itemize}
            \item idに基づき配列内の該当インデックスを特定.
            \item リクエストボディの値でプロパティを上書きする.
            \item 表示: 更新内容をサーバーのターミナルにログ出力する.
            \item 更新後,詳細画面(/stella/:id)を表示する.
        \end{itemize}
\end{itemize}

\subsection{削除機能}
\begin{itemize}
    \item フォーム: GET /stella/check/:id
        \begin{itemize}
            \item 処理: 簡易確認フォームを表示する.
            \item 送信先: GET /stella/delete/:id
        \end{itemize}
    \item 削除処理: GET /stella/delete/:id
        \begin{itemize}
            \item idに基づき配列からspliceで要素を取り除く.
            \item 削除した要素の名前をサーバーのターミナルにログ出力する.
            \item 削除後,一覧画面(/stella)へリダイレクトする.
        \end{itemize}
\end{itemize}
