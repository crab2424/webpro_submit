\chapter[開発者向け仕様書:都道府県一覧表示システム]{\texorpdfstring{開発者向け仕様書:\\都道府県一覧表示システム}{開発者向け仕様書:都道府県一覧表示システム}}

\section{概要}
本仕様書は,Node.jsおよびテンプレートエンジンEJSを用いた「都道府県一覧表示システム」の設計仕様書である.
本システムは,サーバーサイドで都道府県データを管理し,EJSを用いて動的にHTMLを生成・表示する.
データベースの利用は行わず,サーバープロセスのメモリ上(変数)でデータを保持・操作することを前提とする.

\section{データ構造}
サーバー内の配列変数で管理するオブジェクト構造は表\ref{tab:schema}の通りに構成する.

\begin{table}[H]
    \centering
    \caption{都道府県データ構造}
    \label{tab:schema}
    \begin{tabular}{c|c|c}
        \toprule
        プロパティ名 & データ型 & 説明 \\
        \midrule
        id & Number & 一意な識別子 \\
        name & String & 都道府県名 \\
        code & Number & 都道府県番号 \\
        area & Number & 面積($km^2$) \\
        population & Number & 人口(人) \\
        capital & String & 県庁所在地 \\
        region & String & 地方区分 \\
        \bottomrule
    \end{tabular}
\end{table}

\section{ディレクトリ構成}

本システムは,図\ref{fig:pref_dir}に示すディレクトリ構造に従って,使用するファイルを配置する.

\begin{figure}[H]
    \centering
    \begin{verbatim}
webpro_submit/
    |- app/
        |- app_system.js            (メインロジック・データ変数保持)
        |- public/                  (静的ファイル)
            |- index.html           (トップページ)
            |- pref_new.html        (新規作成フォーム)
        |- views/                   (EJSテンプレート)
            |- pref/                (都道府県システム)
                |- pref_check.ejs   (削除確認画面)
                |- pref_detail.ejs  (詳細表示画面)
                |- pref_edit.ejs    (編集フォーム)
                |- pref.ejs         (一覧表示画面)
\end{verbatim}
    \caption{ディレクトリ構成}
    \label{fig:pref_dir}
\end{figure}

\section{HTTPメソッドとルーティング}
本システムにおける各URLとHTTPメソッド,および対応する処理を表\ref{tab:routing}に定義する.

\begin{table}[H]
    \centering
    \caption{ルーティング一覧}
    \label{tab:routing}
    \begin{tabular}{llll}
        \toprule
        機能 & メソッド & パス (URL) & 対応ビュー/処理 \\
        \midrule
        一覧表示 & GET & /pref & views/pref/pref.ejs \\
        新規作成フォーム & GET & /pref/create & public/pref\_new.html \\
        詳細表示 & GET & /pref/:id & views/pref\_detail.ejs \\
        編集フォーム & GET & /pref/edit/:id & views/pref\_edit.ejs \\
        削除確認 & GET & /pref/check/:id & views/pref\_check.ejs \\

        \midrule
        新規データ作成 & POST & /pref & 処理後一覧を表示\\
        新規データ作成& POST & /pref/create & 処理後新規作成へリダイレクト \\
        データ更新 & POST & /pref/update/:id & 処理後詳細ページを表示 \\
        データ削除 & GET & /pref/delete/:id & 処理後一覧へリダイレクト \\
        \bottomrule
    \end{tabular}
\end{table}

\section{ページ遷移}

本システムにおける画面間の遷移を図\ref{fig:flowchart}に示す.なお,本システムのページには戻るリンクを配置するため,一覧表示ページ及び詳細表示ページに直接遷移することが可能である.

\vspace{1cm}
% フローチャート配置場所
\begin{figure}[H]
    \centering
    \includegraphics[width=\textwidth]{fig/dev_pref/flowchart-1.pdf}
    \caption{画面遷移フローチャート}
    \label{fig:flowchart}
\end{figure}

\newpage

\section{各機能・リソース詳細}

\subsection{一覧表示機能}
\begin{itemize}
    \item URL: GET /pref
    \item 処理: サーバー変数の全データをEJSに渡し,for文を用いてテーブル形式でレンダリングする.
    \item 要素: 「新規登録ボタン」,各行ごとの「詳細リンク」.
\end{itemize}

\subsection{新規作成機能}
\begin{itemize}
    \item フォーム (GET /pref/create):
        \begin{itemize}
            \item 表\ref{tab:schema}のidを除くすべてのプロパティを入力フィールドとして表示する.
            \item 送信先は POST /pref.
        \end{itemize}
    \item 作成処理 (POST /pref):
        \begin{itemize}
            \item リクエストボディから値を取得.
            \item 新しいIDを採番し,サーバー変数(配列)にpushする.
            \item 処理完了後,一覧画面(/pref)へリダイレクトする.
        \end{itemize}
\end{itemize}

\subsection{詳細表示機能}
\begin{itemize}
    \item URL: GET /pref/:id
    \item 処理: URLパラメータのIDに基づき配列を検索し,対象データを表示する.
    \item 要素: 「編集ボタン」,「削除ボタン」(formタグによる送信),「一覧に戻るリンク」.
\end{itemize}

\subsection{編集・更新機能}
\begin{itemize}
    \item フォーム (GET /pref/edit/:id):
        \begin{itemize}
            \item 対象データを検索し,value属性に現在の値を埋め込んで表示する.
            \item 送信先は POST /pref/update/:id.
        \end{itemize}
    \item 更新処理 (POST /pref/update/:id):
        \begin{itemize}
            \item IDに基づき配列内の該当インデックスを特定.
            \item リクエストボディの値でプロパティを上書きする.
            \item 詳細画面(/pref/:id)を表示する.
        \end{itemize}
\end{itemize}

\subsection{削除機能}
\begin{itemize}
    \item フォーム (GET /pref/check/:id):
        \begin{itemize}
            \item 簡易確認フォームを表示する.
            \item 確認後、GET /pref/delete/:idで削除処理を実行する.
        \end{itemize}
    \item 削除処理 (GET /pref/delete/:id):
        \begin{itemize}
            \item IDに基づき配列からspliceで要素を取り除く.
            \item 削除完了後,一覧画面(/pref)へリダイレクトする.
        \end{itemize}
\end{itemize}
